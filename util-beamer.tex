%%% custom beamer template  %%%%%%%%%%%%%%%%%%%%%%%%%%%%%%%%%%
\usepackage{beamerthemesplit} % new 
\usepackage{eso-pic}
\usetheme{Berlin}
\usecolortheme{beaver}
\usepackage{transparent}
\xdefinecolor{darkgreen}{rgb}{0,0.60,0}

\newcommand\logoLeftFoot{beamer/logo/ARCnotext.png}
%\newcommand\logoRightFoot{beamer/logo/wpi.png}

\titlegraphic{
        \def\wLogo{2cm}
        \raisebox{-0.5\height}{
        \includegraphics[width=0.4\linewidth,height=0.2\linewidth]{\logoLeftFoot}
        }
        \hspace*{0.2cm}
        \raisebox{-0.5\height}{
        \includegraphics[width=0.4\linewidth,height=0.15\linewidth]{\logoRightFoot}
        }

        %\hspace*{0.2cm}
        %\includegraphics[width=\wLogo,height=\wLogo]{img/logo/INPT.jpg}
}

\tikzset{zlevel/.style={%
execute at begin scope={\pgfonlayer{#1}},
execute at end scope={\endpgfonlayer}
}}


\usetikzlibrary{positioning}

\tikzset{onslide/.code args={<#1>#2}{%
  \only<#1>{\pgfkeysalso{#2}} % \pgfkeysalso doesn't change the path
}}
\tikzset{temporal/.code args={<#1>#2#3#4}{%
  \temporal<#1>{\pgfkeysalso{#2}}{\pgfkeysalso{#3}}{\pgfkeysalso{#4}} % \pgfkeysalso doesn't change the path
}}
\def\R{ \ensuremath{\mathcal{R}} }
\def\C{ \ensuremath{\mathcal{C}} }


\setbeamercolor{alerted text}{fg=orange}
\setbeamercolor{background canvas}{bg=white}
\setbeamercolor{block body alerted}{bg=normal text.bg!90!black}
\setbeamercolor{block body}{bg=gray!10,fg=black}
\setbeamercolor{block body example}{bg=normal text.bg!90!black}
\setbeamercolor{block title}{bg=gray!40!white,fg=black}
\setbeamerfont{block title}{size=\large}

\setbeamercolor{fine separation line}{}
\setbeamercolor{frametitle}{fg=black,bg=gray!30}
\setbeamercolor{normal text}{bg=black,fg=black}
\setbeamercolor{palette sidebar primary}{use=normal text,fg=normal text.fg}
\setbeamercolor{palette sidebar quaternary}{use=structure,fg=structure.fg}
\setbeamercolor{palette sidebar secondary}{use=structure,fg=structure.fg}
\setbeamercolor{palette sidebar tertiary}{use=normal text,fg=normal text.fg}
\setbeamercolor{section in sidebar}{fg=black}
\setbeamercolor{section in sidebar shaded}{fg= grey}
\setbeamercolor{separation line}{}
\setbeamercolor{sidebar}{bg=red}
\setbeamercolor{sidebar}{parent=palette primary}
\setbeamercolor{structure}{bg=black, fg=red}
\setbeamercolor{subsection in sidebar}{fg=black}
\setbeamercolor{subsection in sidebar shaded}{fg=grey}
\setbeamercolor{title}{fg=black!100,bg=gray!30}
\setbeamercolor{titlelike}{fg=brown}
\setbeamercolor{bibliography item}{fg=black}
\setbeamercolor{bibliography entry author}{fg=black}
\setbeamercolor{bibliography entry journal}{fg=black}
\setbeamercolor*{bibliography entry title}{fg=black}

\setbeamertemplate{itemize item}[default]  
\setbeamercolor{enumerate item}{fg=black}  
\setbeamercolor{itemize item}{fg=black}  
\setbeamercolor{item projected}{bg=gray!30!white,fg=white}
\setbeamercolor{subitem projected}{bg=gray!30!white,fg=white}
\setbeamertemplate{itemize subitem}[default]  
\setbeamercolor{itemize subitem}{fg=black!50}  

%several template parameters
\setbeamertemplate{frametitle}[default][center]
\setbeamertemplate{blocks}[rounded][shadow=true]
%\beamertemplateshadingbackground{white!20}{black!20}
\setbeamertemplate{background canvas}[vertical shading][bottom=white,top=white!25]
\setbeamertemplate{sidebar canvas left}[horizontal shading][left=white!40!black,right=black]
\setbeamercolor{footline in sidebar}{fg=black}%[page number]
%\setbeamertemplate{footline}[frame number]

%suppress navigational bar
\beamertemplatenavigationsymbolsempty
%experimental stuff

%\setbeamerfont{footline}{size=\fontsize{10}{12}\selectfont}




%%% tikzstyles %%%%%%%%%%%%%%%%%%%%%%%%%%%%%%%%%%
\tikzstyle{taskRect}=[draw, color=black!70, fill=black!7, rectangle, rounded
corners, thick, minimum width=3cm, minimum height=0.6cm]
\tikzset{
        taskRectH/.style={taskRect, alt=<#1>{color=black!100, fill=black!25, thick, rounded corners}{}, anchor=base},
        taskRectH/.default=+,
        cSphereH/.style={cSphere, alt=<#1>{color=black!100, fill=black!25, thick, rounded corners}{}, anchor=base},
        cSphereH/.default=+,
}

\tikzstyle{borderRect}=[draw, color=red!70, fill=gray!7, rectangle, rounded
corners, thick, minimum width=8cm, minimum height=3cm]
\tikzstyle{testBRect}=[draw, color=darkgreen!100, fill=gray!7, rectangle, rounded
corners, thick, minimum width=8cm, minimum height=3cm]
\tikzstyle{stdBR}=[draw, color=black!100, fill=gray!7, rectangle, rounded
corners, thick, minimum width=8cm, minimum height=3cm]
\tikzstyle{boundingBox}=[draw, color=red!100, fill=gray!7, rectangle, rounded
corners, thick, minimum width=3cm, minimum height=1cm]
\tikzstyle{arrow}=[->, draw, thick]
\tikzstyle{darrow}=[->, draw, ultra thick]
\tikzstyle{uarrow}=[-, draw, ultra thick]
\tikzstyle{hlBox}=[draw, color=red!70, fill=black!7, rectangle, rounded
corners, ultra thick, minimum width=3cm, minimum height=0.6cm]
\tikzstyle{nUnit}=[draw, color=black!70, fill=black!7, circle, rounded
corners, ultra thick, minimum width=1cm, minimum height=0.2cm]
%%%%%%%%%%%%%%%%%%%%%%%%%%%%%%%%%%%%%%%%%%%%%%%%%
\newcommand\framepicture[1]{
\begin{center}
                \begin{tikzpicture}
                        \node[taskRect] at (0,3) {
                        \begin{minipage}[t][4cm]{7cm}
                \includegraphics[width=1.0\linewidth, height=0.9\linewidth]{#1} 
                \end{minipage}
                };
        \end{tikzpicture}
\end{center}
}
%%%%%%%%%%%%%%%%%%%%%%%%%%%%%%%%%%%%%%%%%%%%%%%%%
\newcommand\drawDirectedGraph[2]{
\node[nUnit] (h) at (#2,#1) {h};
\node[nUnit] (v1) at (#2-1,#1-2) {v1};
\node[nUnit] (v2) at (#2+1,#1-2) {v2};
\draw[darrow] (v1) -- (h);
\draw[darrow] (v2) -- (h);
\draw[darrow] (h) -- (#2,#1+1);
}
\newcommand\drawUndirectedGraph[2]{
        \node[nUnit] (uh) at (#2,#1) {h};
        \node[nUnit] (uv1) at (#2-1,#1-2) {v1};
        \node[nUnit] (uv2) at (#2+1,#1-2) {v2};
        \draw[uarrow] (uv1) -- (uh);
        \draw[uarrow] (uv2) -- (uh);
}
\newcommand\drawRBM[2]{
        \node[nUnit] (uh0) at (#2-1.5,#1) {h0};
        \node[nUnit] (uh1) at (#2,#1) {h1};
        \node[nUnit] (uh2) at (#2+1.5,#1) {h2};
        \node[nUnit] (uv1) at (#2-1,#1-2) {v1};
        \node[nUnit] (uv2) at (#2+1,#1-2) {v2};
        \draw[uarrow] (uv1) -- (uh0);
        \draw[uarrow] (uv1) -- (uh1);
        \draw[uarrow] (uv1) -- (uh2);
        \draw[uarrow] (uv2) -- (uh0);
        \draw[uarrow] (uv2) -- (uh1);
        \draw[uarrow] (uv2) -- (uh2);
}
%%%%%%%%%%%%%%%%%%%%%%%%%%%%%%%%%%%%%%%%%%%%%%%%%
\newcommand\frametikzboxthree[4]{
\begin{center}
        \begin{tikzpicture}
                \node[stdBR] at (2,3){
                \begin{minipage}[t][6cm]{10cm} 
                        #1
                        \vfill
                \end{minipage}
                };
                \node[taskRect] at (2,4.5) {#2};
                \pause
                \node[taskRect] at (2,3) {#3};
                \pause
                \node[taskRect] at (2,1.5) {#4};
        \end{tikzpicture}
\end{center}
}
%%%%%%%%%%%%%%%%%%%%%%%%%%%%%%%%%%%%%%%%%%%%%%%%%
\newcommand\frametikzboxfour[5]{
\begin{center}
        \begin{tikzpicture}
                \node[stdBR] at (2,3){
                \begin{minipage}[t][6cm]{10cm} 
                        #1
                        \vfill
                \end{minipage}
                };
                \node[taskRect] at (2,4.5) {#2};
                \pause
                \node[taskRect] at (2,3.5) {#3};
                \pause
                \node[taskRect] at (2,2.5) {#4};
                \pause
                \node[taskRect] at (2,1.5) {#5};
        \end{tikzpicture}
\end{center}
}
\newcommand\frametikzboxtwo[3]{
\begin{center}
        \begin{tikzpicture}
                \node[stdBR] at (2,3){
                \begin{minipage}[t][4cm]{9cm} 
                        #1
                        \vfill
                \end{minipage}
                };
                \node[taskRect] at (2,3.5) {#2};
                \pause
                \node[taskRect] at (2,2) {#3};
        \end{tikzpicture}
\end{center}
}
\useoutertheme{split}
\setbeamertemplate{navigation symbols}{}
%%%%%%%%%%%%%%%%%%%%%%%%%%%%%%%%%%%%%%%%%%%%%%%%%%%%%%%%%%%%%%%%%%%%%%%%%%%%%%%
%% TITLE PAGE
%%%%%%%%%%%%%%%%%%%%%%%%%%%%%%%%%%%%%%%%%%%%%%%%%%%%%%%%%%%%%%%%%%%%%%%%%%%%%%%
\setbeamertemplate{title page}[default][rounded=true,shadow=true]

%\pgfdeclareimage[width=\paperwidth,height=\paperheight]{titlepage}{img/Wanderer_above_the_sea_Caspar_David_Friedrich.jpg}
\defbeamertemplate*{title page}{customized}[1][rounded=true,shadow=true]
{
        %\usebeamerfont{title}\inserttitle\par
        %\usebeamerfont{subtitle}\usebeamercolor[fg]{subtitle}\insertsubtitle\par
        %\bigskip
        %\usebeamerfont{author}\insertauthor\par
        %\usebeamerfont{institute}\insertinstitute\par
        %\usebeamerfont{date}\insertdate\par
        %\usebeamercolor[fg]{titlegraphic}\inserttitlegraphic
        %\includegraphics[width=\paperwidth,height=0.8\paperheight]
        %{img/Wanderer_above_the_sea_Caspar_David_Friedrich.jpg}
        \begin{picture}(0,0)

            \put(-0.1,-0.51){
                    %\pgfuseimage{titlepage}
            }
            \put(-40,0.0){%

                \begin{tikzpicture}
                        \node[taskRect] at (0,0) {
                                \hspace*{1cm}
                        \begin{minipage}[45mm]{10cm}
                                \centering\LARGE \inserttitle
                                \usebeamerfont{subtitle}\usebeamercolor[fg]{subtitle}\insertsubtitle\par
                        \end{minipage}
                        };
                        
                \end{tikzpicture}
            }

            \put(130,-100.5){%

                \begin{tikzpicture}
                        \node[taskRect] at (0,0) {
                        \begin{minipage}[75mm]{8cm}
                                \centering
                                \usebeamerfont{author}\insertauthor\par
                                \usebeamerfont{institute}\insertinstitute\par
                                \usebeamerfont{date}\insertdate\par
                        \end{minipage}
                        %\begin{minipage}[45mm]{9cm}
                                %\usebeamercolor[fg]{titlegraphic}\inserttitlegraphic
                        %\end{minipage}
                        };
                        
                \end{tikzpicture}
            }

    \end{picture}
}

%%%%%%%%%%%%%%%%%%%%%%%%%%%%%%%%%%%%%%%%%%%%%%%%%%%%%%%%%%%%%%%%%%%%%%%%%%%%%%%
%% HEADLINE
%%%%%%%%%%%%%%%%%%%%%%%%%%%%%%%%%%%%%%%%%%%%%%%%%%%%%%%%%%%%%%%%%%%%%%%%%%%%%%%
\setbeamertemplate{headline}{%
\leavevmode%
  \hbox{%
    \begin{beamercolorbox}[wd=\paperwidth,ht=2.5ex,dp=1.125ex]{palette quaternary}%
    \insertsectionnavigationhorizontal{\paperwidth}{}{\hskip0pt plus1filll}
    \end{beamercolorbox}%
  }
}
%%%%%%%%%%%%%%%%%%%%%%%%%%%%%%%%%%%%%%%%%%%%%%%%%%%%%%%%%%%%%%%%%%%%%%%%%%%%%%%
%% FOOTLINE
%%%%%%%%%%%%%%%%%%%%%%%%%%%%%%%%%%%%%%%%%%%%%%%%%%%%%%%%%%%%%%%%%%%%%%%%%%%%%%%
\setbeamertemplate{footline}
{
  \leavevmode%
  \hbox{%
        \def\heightF{4ex}
        \def\depthF{2.5ex}
        \begin{beamercolorbox}[wd=.20\paperwidth,rounded=5cm,ht=\heightF,dp=\depthF,center]{author in head/foot}%
                %\raisebox{-0.35\height}{\includegraphics[width=1.3cm]{\logoLeftFoot}}
                \hspace*{0.2cm}
                \usebeamerfont{author in head/foot}\insertshortauthor
        \end{beamercolorbox}%
        \begin{beamercolorbox}[wd=.60\paperwidth,ht=\heightF,dp=\depthF,center]{title in head/foot}%
                \hspace*{0.3cm}\usebeamerfont{title in head/foot}\insertshorttitle\hspace*{3em}
                %\insertframenumber{} / \inserttotalframenumber\hspace*{1ex}
        \end{beamercolorbox}%
        \begin{beamercolorbox}[wd=.2\paperwidth,ht=\heightF,dp=\depthF,center]{date in head/foot}%
                \insertframenumber{} / \inserttotalframenumber{}
                \hspace*{0.05cm}
                %\raisebox{-0.35\height}{\includegraphics[height=0.45cm,width=1.2cm]{\logoRightFoot}}
        \end{beamercolorbox}}
        \vskip0pt%
}


\newcommand\xcite[2]{
        \begin{center}
        \footnotesize
        {\color{black!75}
        [*]\textit{"#1"}\\
        --- #2}
        \end{center}
}

\newcommand\mycite[2]{
        \begin{center}
        \footnotesize
        [*]\textit{"#1"}\\
        --- #2
        \end{center}
}

\newcommand\headline[1]{
                
        \begin{center}
                \begin{tikzpicture}
                        \node[taskRect,minimum width=9cm,minimum height=3cm] at
                        (0,0) {\Huge #1};
                \end{tikzpicture}
        \end{center}
}
